\subsubsection{fail-else1.bl}
\begin{lstlisting}
else () {}\end{lstlisting}
\subsubsection{fail-elseif1.bl}
\begin{lstlisting}
elseif () {}\end{lstlisting}
\subsubsection{fail-for1.bl}
\begin{lstlisting}
for (;;)\end{lstlisting}
\subsubsection{fail-for2.bl}
\begin{lstlisting}
for () {}\end{lstlisting}
\subsubsection{fail-for3.bl}
\begin{lstlisting}
for {}\end{lstlisting}
\subsubsection{fail-func.bl}
\begin{lstlisting}
def foo() {
  return [3];
}
def call(f) {
  return f();
}
print(call(foo));\end{lstlisting}
\subsubsection{fail-function1.bl}
\begin{lstlisting}
def foo {
  ;
}\end{lstlisting}
\subsubsection{fail-function2.bl}
\begin{lstlisting}
def () {
  ;
}\end{lstlisting}
\subsubsection{fail-function3.bl}
\begin{lstlisting}
def foo()\end{lstlisting}
\subsubsection{fail-generators.bl}
\begin{lstlisting}
Zero\end{lstlisting}
\subsubsection{fail-graph1.bl}
\begin{lstlisting}
G = {
  0->;
}\end{lstlisting}
\subsubsection{fail-graph2.bl}
\begin{lstlisting}
G = {
  ->1;
}\end{lstlisting}
\subsubsection{fail-graph3.bl}
\begin{lstlisting}
G = {
  0->1
}\end{lstlisting}
\subsubsection{fail-if1.bl}
\begin{lstlisting}
if { ; }\end{lstlisting}
\subsubsection{fail-if2.bl}
\begin{lstlisting}
if ()\end{lstlisting}
\subsubsection{fail-selection1.bl}
\begin{lstlisting}
M[range]\end{lstlisting}
\subsubsection{fail-selection2.bl}
\begin{lstlisting}
M[[0;2], [0:2], a] /* 3 things */\end{lstlisting}
\subsubsection{fail-semiring1.bl}
\begin{lstlisting}
<#katonsNameIsActuallyKatie>;\end{lstlisting}
\subsubsection{fail-while1.bl}
\begin{lstlisting}
while () {} /* boolean is empty */\end{lstlisting}
\subsubsection{fail-while2.bl}
\begin{lstlisting}
while {}\end{lstlisting}
\subsubsection{fail-while3.bl}
\begin{lstlisting}
while ()\end{lstlisting}
\subsubsection{test-assignment\_ops.bl}
\begin{lstlisting}
A = 1;
print(toString(A));
A*=2;
print(toString(A));
A~=2;
print(toString(A));
A@=2;
print(toString(A));
A+=1;
print(toString(A));
A^=2;
print(toString(A));
A:=A;
print(toString(A));
\end{lstlisting}
\subsubsection{test-bfs.bl}
\begin{lstlisting}
def plusColumnReduce(A){
    #_;
    return (+%(A^T))^T;
}

def rangeFromVector(v){
    #logical;
    vlogic = v~1;
    #arithmetic;
    n = plusColumnReduce(vlogic);
    u = Zero(n:1);
    j = 0;
    for (i = 0; i < |v|[0]; i += 1) {
        if (v[i]) {
            u[j] = i;
            j = j + 1;
        }
    }
    return u;
}

def One(size){
    A = Zero(size);
    A[range(size[0]), range(size[1])] = 1;
    return A;
}

def BFS(G, frontier){
    #logical;
    N = |G|[0];
    levels = Zero(N : 1);
    maskedGT = G^T;
    depth = 0;
    while (plusColumnReduce(frontier)) {
        #arithmetic;
        depth = depth + 1;
        #logical;
        levels[rangeFromVector(frontier)] = depth;
        mask = !(frontier)[0, Zero(N:1), N, 1];
        maskedGT = maskedGT @ mask;
        frontier = maskedGT*frontier;
    }
    #arithmetic;
    return levels + One(|levels|)~(-1);
}


// Graph is from Algorithms, Papadimitriou et al., Figure 3.9.a
G = [
  0->1;
  1->2;
  1->3;
  1->4;
  2->5;
  4->1;
  4->5;
  4->6;
  5->2;
  5->7;
  6->7;
  6->9;
  7->10;
  8->6;
  9->8;
  10->11;
  11->9;
  12->0
];
frontier = Zero(|G|[0] : 1);
frontier[0] = 1;

print(toString(BFS(G, frontier)));
\end{lstlisting}
\subsubsection{test-column\_reduce.bl}
\begin{lstlisting}
def plusColumnReduce(A){
    #_;
    return (+%(A^T))^T;
}

def timesColumnReduce(A){
    #_;
    return (*%(A^T))^T;
}
A = [3;2];
print(toString(plusColumnReduce(A)));
print(toString(timesColumnReduce(A)));
\end{lstlisting}
\subsubsection{test-comment.bl}
\begin{lstlisting}
// This is a comment
print(toString(4));
/* So is this
print(toString(3));
Nah, we ain't finished yet!
*/
// Now we're done!
print(toString(2));
\end{lstlisting}
\subsubsection{test-compare-select.bl}
\begin{lstlisting}
print(toString(1 == 1[0]));
\end{lstlisting}
\subsubsection{test-convolution1.bl}
\begin{lstlisting}
A = [1,2,3;
    4,5,6;
    7,8,9];
B = I(2);
C = A~B;
print(toString(C));
\end{lstlisting}
\subsubsection{test-convolution2.bl}
\begin{lstlisting}
#logical;
A = [1,2,3;
    4,0,6;
    0,8,9];
C = A~1;
print(toString(C));
\end{lstlisting}
\subsubsection{test-el\_add.bl}
\begin{lstlisting}
M = [1, 3];
N = [2, 4];
print(toString(M + N));
\end{lstlisting}
\subsubsection{test-el\_mul.bl}
\begin{lstlisting}
M = [1, 3];
N = [2, 4];
print(toString(M @ N));\end{lstlisting}
\subsubsection{test-exp1.bl}
\begin{lstlisting}
M = [1, 2; 3, 4];
b = 2;
print(toString(M^b));\end{lstlisting}
\subsubsection{test-for.bl}
\begin{lstlisting}
for(a = 0; a < 8; a+=1){
    print(toString(a));
}
\end{lstlisting}
\subsubsection{test-func-one.bl}
\begin{lstlisting}
def One(size){
    A = Zero(size);
    oneSize = |A|;
    A[range(oneSize[0]), range(oneSize[1])] = 1;
    return A;
}

size = 3;
print(toString(One(size)));
\end{lstlisting}
\subsubsection{test-func1.bl}
\begin{lstlisting}
def f(M) {
  return M;
}

print(toString(f([3;3])));
\end{lstlisting}
\subsubsection{test-func2.bl}
\begin{lstlisting}
def foo() {
    return [];
}

print(toString(foo()));\end{lstlisting}
\subsubsection{test-func3.bl}
\begin{lstlisting}
def foo() {
    return;
}

print(toString(foo()));
\end{lstlisting}
\subsubsection{test-func4.bl}
\begin{lstlisting}
M = [3];

def foo(M) {
  M[0] = [4];
}

foo(M);

print(toString(M));
\end{lstlisting}
\subsubsection{test-func5.bl}
\begin{lstlisting}
M = [3];

def foo(M) {
  M = [4];
}

foo(M);

print(toString(M));
\end{lstlisting}
\subsubsection{test-func6.bl}
\begin{lstlisting}
def a(G){
    G = G + 1;
}

a(1);
\end{lstlisting}
\subsubsection{test-generator1.bl}
\begin{lstlisting}
print(toString(Zero([4])));
print(toString(Zero([3;2])));\end{lstlisting}
\subsubsection{test-generator2.bl}
\begin{lstlisting}
print(toString(I(3)));\end{lstlisting}
\subsubsection{test-generator3.bl}
\begin{lstlisting}
print(toString(range(3)));
print(toString(range([-2; 2])));
\end{lstlisting}
\subsubsection{test-graph1.bl}
\begin{lstlisting}
G = [
  0->1
];
print(toString(G));\end{lstlisting}
\subsubsection{test-graph2.bl}
\begin{lstlisting}
G = [
    0->1;
    1->0;
    1->2;
    4->17
];
print(toString(G));\end{lstlisting}
\subsubsection{test-helloworld.bl}
\begin{lstlisting}
print([65;66;67;68]);
\end{lstlisting}
\subsubsection{test-if1.bl}
\begin{lstlisting}
A = [0; 1];
if (A + [1; 1] > [1; 1]) {
  print(toString(A + [1; 2]));
} else {
  print(toString(A + [3; 4]));
}\end{lstlisting}
\subsubsection{test-if2.bl}
\begin{lstlisting}
A = 0;
if (A) {
  A = 0;
} else {
  A = 3;
}
if (A) {
  A = 1;
} else {
  A = 3;
}
print(toString(A));
\end{lstlisting}
\subsubsection{test-local1.bl}
\begin{lstlisting}
M = [65,66;67,68];

print(toString(M));
\end{lstlisting}
\subsubsection{test-local2.bl}
\begin{lstlisting}
{A = 3;}
print(toString(A));
\end{lstlisting}
\subsubsection{test-matmul1.bl}
\begin{lstlisting}
A = [1,2;
    1,2;
    1,2;
    1,2];
B = [1,2,3;
    1,2,3];
C = A*B;
print(toString(C));

\end{lstlisting}
\subsubsection{test-matmul2.bl}
\begin{lstlisting}
#logical;
print(toString(5 * 0));
print(toString(5 * 3));
#maxmin;
print(toString(5 * 0));
print(toString(5 * 3));
\end{lstlisting}
\subsubsection{test-neg.bl}
\begin{lstlisting}
A = [1,0;
    0,3];
print(toString(!A));
\end{lstlisting}
\subsubsection{test-print\_return.bl}
\begin{lstlisting}
A = print(toString(5));
\end{lstlisting}
\subsubsection{test-range\_from\_vector.bl}
\begin{lstlisting}
def plusColumnReduce(A){
    #_;
    return (+%(A^T))^T;
}

def rangeFromVector(v){
    #logical;
    vlogic = v~1;
    #arithmetic;
    n = plusColumnReduce(vlogic);
    u = Zero(n:1);
    j = 0;
    for (i = 0; i < |v|[0]; i += 1) {
        if (v[i]) {
            u[j] = i;
            j = j + 1;
        }
    }
    return u;
}

A = rangeFromVector([3;0;1;2;0;5]);
print(toString(A));
\end{lstlisting}
\subsubsection{test-reduce\_rows1.bl}
\begin{lstlisting}
A = [1,2;
    3,4;
    5,6];
B = +%A;
C = *%A;
print(toString(B));
print(toString(C));\end{lstlisting}
\subsubsection{test-reduce\_rows2.bl}
\begin{lstlisting}
#maxmin;
A = [3,6;
    2,4;
    -1,2];
B = +%A;
C = *%A;
print(toString(B));
print(toString(C));
\end{lstlisting}
\subsubsection{test-selection1.bl}
\begin{lstlisting}
M = Zero(4);
M[[0;2], [0;2]] = 1;
print(toString(M));\end{lstlisting}
\subsubsection{test-selection2.bl}
\begin{lstlisting}
M = Zero(2);
M[1, 0] = 1;
N = Zero(3);
N[1, 1, 2, 2] = I(2);

print(toString(M));
print(toString(N));\end{lstlisting}
\subsubsection{test-selection3.bl}
\begin{lstlisting}
v = [1;1;1];
v[1] = 2;
u = [1;1;1];
u[[0;2]] = 2;
print(toString(v));
print(toString(u));\end{lstlisting}
\subsubsection{test-selection4.bl}
\begin{lstlisting}
v = Zero([5;1]);
v[range(5)] = 1;
print(toString(v));
\end{lstlisting}
\subsubsection{test-selection5.bl}
\begin{lstlisting}
A = [1,2,3;4,5,6;7,8,9];
B = A[0,0,2,2];
print(toString(B));\end{lstlisting}
\subsubsection{test-selection6.bl}
\begin{lstlisting}
A = [1,2,3;
    4,5,6;
    7,8,9];
B = A[[0;2], [0;2] , 1, 1];
v = [1;2;3;4];
u = v[[0;2;3]];
print(toString(B));
print(toString(u));\end{lstlisting}
\subsubsection{test-selection7.bl}
\begin{lstlisting}
A = [1,2;2,3][0,0,0,0];
print(toString(A));
\end{lstlisting}
\subsubsection{test-semiring1.bl}
\begin{lstlisting}
def prints(M) {
    print(toString(M));
    return;
}

a = 2;
b = 3;
c = 0;

#arithmetic;
prints(a + b);
prints(a * b);
prints(a * c);

#logical;
prints(a + b);
prints(a * b);
prints(a * c);

#maxmin;
prints(a + b);
prints(a * b);
prints(a * c);
\end{lstlisting}
\subsubsection{test-semiring2.bl}
\begin{lstlisting}
def prints(M) {
    print(toString(M));
    print([10]);
    return;
}

def g1(A, B) {
    #maxmin;
    prints(A * B);
    return;
}

def g2(A, B) {
    prints(A * B);
    return;
}

def g3(A, B) {
    #_;
    prints(A * B);
    return;
}

def f(A, B) {
    #maxmin;
    prints(A * B);
    #logical;
    g1(A, B);
    g2(A, B);
    g3(A, B);
    return;
}

A = [0,1;
     2,3;
     4,5];

B = [0,0,5;
     3,4,0];

prints(A * B);

f(A, B);

/*
    Should be:
        1) arithmetic
        2) maxmin
        3) maxmin
        4) arithmetic
        5) logical
*/
\end{lstlisting}
\subsubsection{test-size.bl}
\begin{lstlisting}
A = [1,2,3;4,5,6];
B = |A|;
print(toString(B));\end{lstlisting}
\subsubsection{test-standardlib.bl}
\begin{lstlisting}
def plusColumnReduce(A){
    #_;
    return (+%(A^T))^T;
}

def rangeFromVector(v){
    #logical;
    vlogic = v~1;
    #arithmetic;
    n = plusColumnReduce(vlogic);
    u = Zero(n:1);
    j = 0;
    i = 0;
    while (i < |v|[0]) {
        if (v[i]) {
            u[j] = i;
            j = j + 1;
        }
        i = i + 1;
    }
    return u;
}

def One(size){
    A = Zero(size);
    oneSize = |A|;
    A[range(oneSize[0]), range(oneSize[1])] = 1;
    return A;
}
\end{lstlisting}
\subsubsection{test-string1.bl}
\begin{lstlisting}
A = 'BLAS';
print(toString(A));
\end{lstlisting}
\subsubsection{test-transpose.bl}
\begin{lstlisting}
M = [1, 2; 3, 4];
print(toString(M^T));\end{lstlisting}
\subsubsection{test-vert\_concat.bl}
\begin{lstlisting}
A = [1,2];
B = [3,4;
    5,6];
C = A:B;
print(toString(C));\end{lstlisting}
\subsubsection{test-while1.bl}
\begin{lstlisting}
A = 1;
B = 10;
while (A < B) {
  A = A + 1;
  print(toString(A));
}
C = 4;
while (C < B) {
  print(toString(C));
  C = C + 1;
}
\end{lstlisting}
\subsubsection{test-while2.bl}
\begin{lstlisting}
def doubler(A) {
  i = 0;
  B = 1;
  while (i < A) {
    B = B @ 2;
    i = i + 1;
  }
  return B;
}

print(toString(doubler(4)));\end{lstlisting}
